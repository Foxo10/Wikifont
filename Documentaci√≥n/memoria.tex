\documentclass[11pt,a4paper]{article}
\usepackage[spanish]{babel}
\selectlanguage{spanish}
\usepackage[T1]{fontenc}
\usepackage{amsmath}
\usepackage{makeidx}
\usepackage[utf8]{inputenc}
\usepackage{lipsum}
\usepackage{graphicx} % figuras
\usepackage{subfigure} % subfiguras
\usepackage{cite} 
\usepackage{indentfirst}
\usepackage{listings}
\usepackage{float}


\begin{document}
\begin{titlepage}
 \vfill
  \begin{center}
   {\Huge\textbf{Universidad del País Vasco}} \\[1.5cm]
   
   % Contenedor para las imágenes
   \begin{minipage}[t]{0.5\textwidth}
   \centering
   \includegraphics[width=\textwidth]{imagenes/upv_logo.jpg}
   \end{minipage}
   \\ % Espacio entre imágenes
   \begin{minipage}[t]{0.5\textwidth}
   \centering
   \includegraphics[width=\textwidth]{imagenes/wikifont logo.png}
   \end{minipage}
   
   \vfill

   {\huge WikiFont}\\[2.5cm]

   \hspace{.45\textwidth}
   \begin{minipage}{.5\textwidth}
   \large Proyecto individual presentado a la disciplina de Desarrollo Avanzado de Software del UPV/EHU.\\[1cm]

Profesor Iker Sobrón Polancos

  \end{minipage}
  \vfill

\vspace{2cm}

\large \textbf{Oier Díez Gutiérrez}\\

\large \textbf{Junio de 2025}
\end{center}
\end{titlepage}


\tableofcontents{}
\newpage

\section{Motivación}


\section{Descripción y objetivo de la \textit{aplicación}}

\subsection{Descripción de las clases}

\begin{itemize}
    \item 
    \item 
    \item 
\end{itemize}

\subsection{Funcionalidades obligatorias y adicionales}

Funcionalidades principales:
\begin{itemize}
  \item  
  \item 
  \item 
  \item 
    
\end{itemize}

	
    Funcionalidades extras:\\
    
\begin{itemize}
  \item 
  \item 
  \item 
  \item 
  \item 
\end{itemize}


\section{Códigos}

Todo el código está disponible en el repositorio de Github: \\ \textit{https://github.com/Foxo10/Wikifont}.



\section{Manual de usuario}

	La figura [\ref{tini}] muestra la primera pantalla de la aplicación, donde ... 
    
    Las figuras [\ref{gpsatv}][\ref{intatv}][\ref{gpswifinocont}] muestran los modos 
    
    La pantalla [\ref{cad}] corresponde al registro de datos del usuario, 
    Tras completar el registro, la pantalla inicial pasa a ser la figura [\ref{connowifiegps}], donde
    
    La figura [\ref{loadcont}] muestra la pantalla previa a la selección de contactos, que corresponde a la pantalla [\ref{selcont}]. 
    La información relacionada con X función se encuentra en la figura [\ref{inf}].



\begin{figure}[H]
    \centering
    \includegraphics[scale = 0.2]{imagenes/1telainicial.png}
    \caption{Pantalla inicial}
    \label{tini}
\end{figure}

\section{Dificultades afrontadas}

    Problemas con la base de datos de Room. Fuente mal añadida en el .csv.
    Dificultades con las interfaces tras implementar temas y estilos personalizados.



\bibliographystyle{abbrv}
%\bibliographystyle{apa}
%\bibliography{referencias}
\begin{thebibliography}{}

	\bibitem{}
    Pictogrammers.
	\newblock Material design icons \\
	\newblock Disponible en \textit{https://pictogrammers.com/library/mdi/}.  
    
    \bibitem{}
    Android: ImageSwitcher.
	\newblock In/Out animation. \\
	\newblock Disponible en \textit{https://stackoverflow.com/questions/5950831/android-imageswitcher}.  
    
    \bibitem{}
    Florina Muntenescu.
	\newblock 7 Pro-tips for Room \\
	\newblock Disponible en \textit{https://medium.com/androiddevelopers/7-pro-tips-for-room-fbadea4bfbd1}.

    \bibitem{}
    Codelabs.
	\newblock Android Room with a View - Java \\
	\newblock Disponible en \textit{https://developer.android.com/codelabs/android-room-with-a-view?authuser=1#8}.
    
    
    

\end{thebibliography}


\end{document}
